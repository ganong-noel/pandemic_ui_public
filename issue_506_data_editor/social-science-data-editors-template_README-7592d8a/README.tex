% Options for packages loaded elsewhere
\PassOptionsToPackage{unicode}{hyperref}
\PassOptionsToPackage{hyphens}{url}
%
\documentclass[
]{article}
\usepackage{lmodern}
\usepackage{amssymb,amsmath}
\usepackage{ifxetex,ifluatex}
\ifnum 0\ifxetex 1\fi\ifluatex 1\fi=0 % if pdftex
  \usepackage[T1]{fontenc}
  \usepackage[utf8]{inputenc}
  \usepackage{textcomp} % provide euro and other symbols
\else % if luatex or xetex
  \usepackage{unicode-math}
  \defaultfontfeatures{Scale=MatchLowercase}
  \defaultfontfeatures[\rmfamily]{Ligatures=TeX,Scale=1}
\fi
% Use upquote if available, for straight quotes in verbatim environments
\IfFileExists{upquote.sty}{\usepackage{upquote}}{}
\IfFileExists{microtype.sty}{% use microtype if available
  \usepackage[]{microtype}
  \UseMicrotypeSet[protrusion]{basicmath} % disable protrusion for tt fonts
}{}
\makeatletter
\@ifundefined{KOMAClassName}{% if non-KOMA class
  \IfFileExists{parskip.sty}{%
    \usepackage{parskip}
  }{% else
    \setlength{\parindent}{0pt}
    \setlength{\parskip}{6pt plus 2pt minus 1pt}}
}{% if KOMA class
  \KOMAoptions{parskip=half}}
\makeatother
\usepackage{xcolor}
\IfFileExists{xurl.sty}{\usepackage{xurl}}{} % add URL line breaks if available
\IfFileExists{bookmark.sty}{\usepackage{bookmark}}{\usepackage{hyperref}}
\hypersetup{
  hidelinks,
  pdfcreator={LaTeX via pandoc}}
\urlstyle{same} % disable monospaced font for URLs
\usepackage{longtable,booktabs}
% Correct order of tables after \paragraph or \subparagraph
\usepackage{etoolbox}
\makeatletter
\patchcmd\longtable{\par}{\if@noskipsec\mbox{}\fi\par}{}{}
\makeatother
% Allow footnotes in longtable head/foot
\IfFileExists{footnotehyper.sty}{\usepackage{footnotehyper}}{\usepackage{footnote}}
\makesavenoteenv{longtable}
\setlength{\emergencystretch}{3em} % prevent overfull lines
\providecommand{\tightlist}{%
  \setlength{\itemsep}{0pt}\setlength{\parskip}{0pt}}
\setcounter{secnumdepth}{-\maxdimen} % remove section numbering

\author{}
\date{}

\begin{document}

\hypertarget{template-readme-and-guidance}{%
\section{Template README and
Guidance}\label{template-readme-and-guidance}}

\begin{quote}
INSTRUCTIONS: This README suggests structure and content that have been
approved by various journals, see
\href{https://social-science-data-editors.github.io/template_README/Endorsers.html}{Endorsers}.
It is available as
\href{https://github.com/social-science-data-editors/template_README/blob/master/template-README.html}{Markdown/txt},
\href{https://social-science-data-editors.github.io/template_README/templates/README.docx}{Word},
\href{https://social-science-data-editors.github.io/template_README/templates/README.tex}{LaTeX},
and
\href{https://social-science-data-editors.github.io/template_README/templates/README.pdf}{PDF}.
In practice, there are many variations and complications, and authors
should feel free to adapt to their needs. All instructions can (should)
be removed from the final README (in Markdown, remove lines starting
with \texttt{\textgreater{}\ INSTRUCTIONS}). Please ensure that a PDF is
submitted in addition to the chosen native format.
\end{quote}

\hypertarget{overview}{%
\subsection{Overview}\label{overview}}

\begin{quote}
INSTRUCTIONS: The typical README in social science journals serves the
purpose of guiding a reader through the available material and a route
to replicating the results in the research paper. Start by providing a
brief overview of the available material and a brief guide as to how to
proceed from beginning to end.
\end{quote}

Example: The code in this replication package constructs the analysis
file from the three data sources (Ruggles et al, 2018; Inglehart et al,
2019; BEA, 2016) using Stata and Julia. Two main files run all of the
code to generate the data for the 15 figures and 3 tables in the paper.
The replicator should expect the code to run for about 14 hours.

\hypertarget{data-availability-and-provenance-statements}{%
\subsection{Data Availability and Provenance
Statements}\label{data-availability-and-provenance-statements}}

\begin{quote}
INSTRUCTIONS: Every README should contain a description of the origin
(provenance), location and accessibility (data availability) of the data
used in the article. These descriptions are generally referred to as
``Data Availability Statements'' (DAS). However, in some cases, there is
no external data used.
\end{quote}

\begin{itemize}
\tightlist
\item[$\square$]
  This paper does not involve analysis of external data (i.e., no data
  are used or the only data are generated by the authors via simulation
  in their code).
\end{itemize}

\begin{quote}
If box above is checked and if no simulated/synthetic data files are
provided by the authors, please skip directly to the section on
\protect\hyperlink{computational-requirements}{Computational
Requirements}. Otherwise, continue.
\end{quote}

\begin{quote}
INSTRUCTIONS: - When the authors are \textbf{secondary data users} (they
did not generate the data), the provenance and DAS coincide, and should
describe the condition under which (a) the current authors (b) any
future users might access the data. - When the data were generated (by
the authors) in the course of conducting (lab or field)
\textbf{experiments}, or were collected as part of \textbf{surveys},
then the description of the provenance should describe the data
generating process, i.e., survey or experimental procedures: -
Experiments: complete sets of experimental instructions, questionnaires,
stimuli for all conditions, potentially screenshots, scripts for
experimenters or research assistants, as well as for subject eligibility
criteria (e.g.~selection criteria, exclusions), recruitment waves,
demographics of subject pool used. - For lab experiments specifically, a
description of any pilot sessions/studies, and computer programs,
configuration files, or scripts used to run the experiment. - For
surveys, the whole questionnaire (code or images/PDF) including survey
logic if not linear, interviewer instructions, enumeration lists, sample
selection criteria.

The information should describe ALL data used, regardless of whether
they are provided as part of the replication archive or not, and
regardless of size or scope. The DAS should provide enough information
that a replicator can obtain the data from the original source, even if
the file is provided.

For instance, if using GDP deflators, the source of the deflators
(e.g.~at the national statistical office) should also be listed here. If
any of this information has been provided in a pre-registration, then a
link to that registration may (partially) suffice.

DAS can be complex and varied. Examples are provided
\href{https://social-science-data-editors.github.io/guidance/Requested_information_dcas.html}{here},
and below.

Importantly, if providing the data as part of the replication package,
authors should be clear about whether they have the \textbf{rights} to
distribute the data. Data may be subject to distribution restrictions
due to sensitivity, IRB, proprietary clauses in the data use agreement,
etc.

NOTE: DAS do not replace Data Citations (see
\href{https://social-science-data-editors.github.io/template_README/Data_citation_guidance.html}{Guidance}).
Rather, they augment them. Depending on journal requirements and to some
extent stylistic considerations, data citations should appear in the
main article, in an appendix, or in the README. However, data citations
only provide information \textbf{where} to find the data, not
\textbf{how to access} those data. Thus, DAS augment data citations by
going into additional detail that allow a researcher to assess cost,
complexity, and availability over time of the data used by the original
author.
\end{quote}

\hypertarget{statement-about-rights}{%
\subsubsection{Statement about Rights}\label{statement-about-rights}}

\begin{itemize}
\tightlist
\item[$\square$]
  I certify that the author(s) of the manuscript have legitimate access
  to and permission to use the data used in this manuscript.
\item[$\square$]
  I certify that the author(s) of the manuscript have documented
  permission to redistribute/publish the data contained within this
  replication package. Appropriate permission are documented in the
  \href{https://social-science-data-editors.github.io/template_README/LICENSE.txt}{LICENSE.txt}
  file.
\end{itemize}

\hypertarget{optional-but-recommended-license-for-data}{%
\subsubsection{(Optional, but recommended) License for
Data}\label{optional-but-recommended-license-for-data}}

\begin{quote}
INSTRUCTIONS: Most data repositories provide for a default license, but
do not impose a specific license. Authors should actively select a
license. This should be provided in a LICENSE.txt file, separately from
the README, possibly combined with the license for any code. Some data
may be subject to inherited license requirements, i.e., the data
provider may allow for redistribution only if the data is licensed under
specific rules - authors should check with their data providers. For
instance, a data use license might require that users - the current
author, but also any subsequent users - cite the data provider.
Licensing can be complex. Some non-legal guidance may be found
\href{https://social-science-data-editors.github.io/guidance/Licensing_guidance.html}{here}.
For multiple licenses within a data package, the \texttt{LICENSE.txt}
file might contain the concatenation of all the licenses that apply (for
instance, a custom license for one file, plus a CC-BY license for
another file).

NOTE: In many cases, it is not up to the creator of the replication
package to simply define a license, a license may be \emph{sticky} and
be defined by the original data creator.
\end{quote}

\emph{Example:} The data are licensed under a Creative Commons/CC-BY-NC
license. See LICENSE.txt for details.

\hypertarget{summary-of-availability}{%
\subsubsection{Summary of Availability}\label{summary-of-availability}}

\begin{itemize}
\tightlist
\item[$\square$]
  All data \textbf{are} publicly available.
\item[$\square$]
  Some data \textbf{cannot be made} publicly available.
\item[$\square$]
  \textbf{No data can be made} publicly available.
\end{itemize}

\hypertarget{details-on-each-data-source}{%
\subsubsection{Details on each Data
Source}\label{details-on-each-data-source}}

\begin{quote}
INSTRUCTIONS: For each data source, list the file that contains data
from that source here; if providing combined/derived datafiles, list
them separately after the DAS. For each data source or file, as
appropriate,

\begin{itemize}
\tightlist
\item
  Describe the format (open formats preferred, but some
  software-specific formats OK if open-source readers available):
  \texttt{.dta}, \texttt{.xlsx}, \texttt{.csv}, \texttt{netCDF}, etc.
\item
  Provide a data dictionairy, either as part of the archive (list the
  file name), or at a URL (list the URL). Some formats are
  self-describing \emph{if} they have the requisite information (e.g.,
  \texttt{.dta} should have both variable and value labels).
\item
  List availability within the package
\item
  Use proper bibliographic references in addition to a verbose
  description (and provide a bibliography at the end of the README,
  expanding those references)
\end{itemize}

A summary in tabular form can be useful:
\end{quote}

\begin{longtable}[]{@{}lllll@{}}
\toprule
\begin{minipage}[b]{0.17\columnwidth}\raggedright
Data.Name\strut
\end{minipage} & \begin{minipage}[b]{0.17\columnwidth}\raggedright
Data.Files\strut
\end{minipage} & \begin{minipage}[b]{0.17\columnwidth}\raggedright
Location\strut
\end{minipage} & \begin{minipage}[b]{0.17\columnwidth}\raggedright
Provided\strut
\end{minipage} & \begin{minipage}[b]{0.17\columnwidth}\raggedright
Citation\strut
\end{minipage}\tabularnewline
\midrule
\endhead
\begin{minipage}[t]{0.17\columnwidth}\raggedright
``Current Population Survey 2018''\strut
\end{minipage} & \begin{minipage}[t]{0.17\columnwidth}\raggedright
cepr\_march\_2018.dta\strut
\end{minipage} & \begin{minipage}[t]{0.17\columnwidth}\raggedright
data/\strut
\end{minipage} & \begin{minipage}[t]{0.17\columnwidth}\raggedright
TRUE\strut
\end{minipage} & \begin{minipage}[t]{0.17\columnwidth}\raggedright
CEPR (2018)\strut
\end{minipage}\tabularnewline
\begin{minipage}[t]{0.17\columnwidth}\raggedright
``Provincial Administration Reports''\strut
\end{minipage} & \begin{minipage}[t]{0.17\columnwidth}\raggedright
coast\_simplepoint2.csv; rivers\_simplepoint2.csv; RAIL\_dummies.dta;
railways\_Dissolve\_Simplify\_point2.csv\strut
\end{minipage} & \begin{minipage}[t]{0.17\columnwidth}\raggedright
Data/maps/\strut
\end{minipage} & \begin{minipage}[t]{0.17\columnwidth}\raggedright
TRUE\strut
\end{minipage} & \begin{minipage}[t]{0.17\columnwidth}\raggedright
Administration (2017)\strut
\end{minipage}\tabularnewline
\begin{minipage}[t]{0.17\columnwidth}\raggedright
``2017 SAT scores''\strut
\end{minipage} & \begin{minipage}[t]{0.17\columnwidth}\raggedright
Not available\strut
\end{minipage} & \begin{minipage}[t]{0.17\columnwidth}\raggedright
data/to\_clean/\strut
\end{minipage} & \begin{minipage}[t]{0.17\columnwidth}\raggedright
FALSE\strut
\end{minipage} & \begin{minipage}[t]{0.17\columnwidth}\raggedright
College Board (2020)\strut
\end{minipage}\tabularnewline
\bottomrule
\end{longtable}

where the \texttt{Data.Name} column is then expanded in the subsequent
paragraphs, and \texttt{CEPR\ (2018)} is resolved in the References
section of the README.

\hypertarget{example-for-public-use-data-collected-by-the-authors}{%
\subsubsection{Example for public use data collected by the
authors}\label{example-for-public-use-data-collected-by-the-authors}}

\begin{quote}
The {[}DATA TYPE{]} data used to support the findings of this study have
been deposited in the {[}NAME{]} repository ({[}DOI or OTHER PERSISTENT
IDENTIFIER{]}).
{[}\href{https://www.hindawi.com/research.data/\#statement.templates}{1}{]}.
The data were collected by the authors, and are available under a
Creative Commons Non-commercial license.
\end{quote}

\hypertarget{example-for-public-use-data-sourced-from-elsewhere-and-provided}{%
\subsubsection{Example for public use data sourced from elsewhere and
provided}\label{example-for-public-use-data-sourced-from-elsewhere-and-provided}}

\begin{quote}
Data on National Income and Product Accounts (NIPA) were downloaded from
the U.S. Bureau of Economic Analysis (BEA, 2016). We use Table 30. Data
can be downloaded from https://apps.bea.gov/regional/downloadzip.cfm,
under ``Personal Income (State and Local)'', select CAINC30: Economic
Profile by County, then download. Data can also be directly downloaded
using https://apps.bea.gov/regional/zip/CAINC30.zip. A copy of the data
is provided as part of this archive. The data are in the public domain.
\end{quote}

Datafile: \texttt{CAINC30\_\_ALL\_AREAS\_1969\_2018.csv}

\hypertarget{example-for-public-use-data-with-required-registration-and-provided-extract}{%
\subsubsection{Example for public use data with required registration
and provided
extract}\label{example-for-public-use-data-with-required-registration-and-provided-extract}}

\begin{quote}
The paper uses IPUMS Terra data (Ruggles et al, 2018). IPUMS-Terra does
not allow for redistribution, except for the purpose of replication
archives. Permissions as per https://terra.ipums.org/citation have been
obtained, and are documented within the ``data/IPUMS-terra'' folder.
\textgreater{} Note: the reference to ``Ruggles et al, 2018'' would be
resolved in the Reference section of this README, \textbf{and} in the
main manuscript.
\end{quote}

Datafile: \texttt{data/raw/ipums\_terra\_2018.dta}

\hypertarget{example-for-free-use-data-with-required-registration-extract-not-provided}{%
\subsubsection{Example for free use data with required registration,
extract not
provided}\label{example-for-free-use-data-with-required-registration-extract-not-provided}}

\begin{quote}
The paper uses data from the World Values Survey Wave 6 (Inglehart et
al, 2019). Data is subject to a redistribution restriction, but can be
freely downloaded from
http://www.worldvaluessurvey.org/WVSDocumentationWV6.jsp. Choose
\texttt{WV6\_Data\_Stata\_v20180912}, fill out the registration form,
including a brief description of the project, and agree to the
conditions of use. Note: ``the data files themselves are not
redistributed'' and other conditions. Save the file in the directory
\texttt{data/raw}.
\end{quote}

\begin{quote}
\begin{quote}
Note: the reference to ``Inglehart et al, 2018'' would be resolved in
the Reference section of this README, \textbf{and} in the main
manuscript.
\end{quote}
\end{quote}

Datafile: \texttt{data/raw/WV6\_Data\_Stata\_v20180912.dta} (not
provided)

\hypertarget{example-for-confidential-data}{%
\subsubsection{Example for confidential
data}\label{example-for-confidential-data}}

\begin{quote}
INSTRUCTIONS: Citing and describing confidential data, in particular
when it does not have a regular distribution channel or online landing
page, can be tricky. A citation can be crafted
(\href{https://social-science-data-editors.github.io/guidance/FAQ.html\#data-citation-without-online-link}{see
guidance}), and the DAS should describe how to access, whom to contact
(including the role of the particular person, should that person
retire), and other relevant information, such as required citizenship
status or cost.
\end{quote}

\begin{quote}
The data for this project (DESE, 2019) are confidential, but may be
obtained with Data Use Agreements with the Massachusetts Department of
Elementary and Secondary Education (DESE). Researchers interested in
access to the data may contact {[}NAME{]} at {[}EMAIL{]}, also see
www.doe.mass.edu/research/contact.html. It can take some months to
negotiate data use agreements and gain access to the data. The author
will assist with any reasonable replication attempts for two years
following publication.
\end{quote}

\hypertarget{example-for-confidential-census-bureau-data}{%
\subsubsection{Example for confidential Census Bureau
data}\label{example-for-confidential-census-bureau-data}}

\begin{quote}
All the results in the paper use confidential microdata from the U.S.
Census Bureau. To gain access to the Census microdata, follow the
directions here on how to write a proposal for access to the data via a
Federal Statistical Research Data Center:
https://www.census.gov/ces/rdcresearch/howtoapply.html. You must request
the following datasets in your proposal: 1. Longitudinal Business
Database (LBD), 2002 and 2007 2. Foreign Trade Database -- Import (IMP),
2002 and 2007 {[}\ldots{]}
\end{quote}

(adapted from \href{https://doi.org/10.1093/restud/rdw057}{Fort (2016)})

\hypertarget{example-for-preliminary-code-during-the-editorial-process}{%
\subsubsection{Example for preliminary code during the editorial
process}\label{example-for-preliminary-code-during-the-editorial-process}}

\begin{quote}
Code for data cleaning and analysis is provided as part of the
replication package. It is available at
https://dropbox.com/link/to/code/XYZ123ABC for review. It will be
uploaded to the {[}JOURNAL REPOSITORY{]} once the paper has been
conditionally accepted.
\end{quote}

\hypertarget{dataset-list}{%
\subsection{Dataset list}\label{dataset-list}}

\begin{quote}
INSTRUCTIONS: In some cases, authors will provide one dataset (file) per
data source, and the code to combine them. In others, in particular when
data access might be restrictive, the replication package may only
include derived/analysis data. Every file should be described. This can
be provided as a Excel/CSV table, or in the table below.
\end{quote}

\begin{quote}
INSTRUCTIONS: While it is often most convenient to provide data in the
native format of the software used to analyze and process the data, not
all formats are ``open'' and can be read by other (free) software. Data
should at a minimum be provided in formats that can be read by
open-source software (R, Python, others), and ideally be provided in
non-proprietary, archival-friendly formats.
\end{quote}

\begin{quote}
INSTRUCTIONS: All data files should be fully documented:
variables/columns should have labels (long-form meaningful names), and
values should be explained. This might mean generating a codebook,
pointing at a public codebook, or providing data in (non-proprietary)
formats that allow for a rich description. This is in particular
important for data that is not distributable.
\end{quote}

\begin{quote}
INSTRUCTIONS: Some journals require, and it is considered good practice,
to provide synthetic or simulated data that has some of the key
characteristics of the restricted-access data which are not provided.
The level of fidelity may vary - it may be useful for debugging only, or
it should allow to assess the key characteristics of the
statistical/econometric procedure or the main conclusions of the paper.
\end{quote}

\begin{longtable}[]{@{}llll@{}}
\toprule
\begin{minipage}[b]{0.26\columnwidth}\raggedright
Data file\strut
\end{minipage} & \begin{minipage}[b]{0.19\columnwidth}\raggedright
Source\strut
\end{minipage} & \begin{minipage}[b]{0.23\columnwidth}\raggedright
Notes\strut
\end{minipage} & \begin{minipage}[b]{0.21\columnwidth}\raggedright
Provided\strut
\end{minipage}\tabularnewline
\midrule
\endhead
\begin{minipage}[t]{0.26\columnwidth}\raggedright
\texttt{data/raw/lbd.dta}\strut
\end{minipage} & \begin{minipage}[t]{0.19\columnwidth}\raggedright
LBD\strut
\end{minipage} & \begin{minipage}[t]{0.23\columnwidth}\raggedright
Confidential\strut
\end{minipage} & \begin{minipage}[t]{0.21\columnwidth}\raggedright
No\strut
\end{minipage}\tabularnewline
\begin{minipage}[t]{0.26\columnwidth}\raggedright
\texttt{data/raw/terra.dta}\strut
\end{minipage} & \begin{minipage}[t]{0.19\columnwidth}\raggedright
IPUMS Terra\strut
\end{minipage} & \begin{minipage}[t]{0.23\columnwidth}\raggedright
As per terms of use\strut
\end{minipage} & \begin{minipage}[t]{0.21\columnwidth}\raggedright
Yes\strut
\end{minipage}\tabularnewline
\begin{minipage}[t]{0.26\columnwidth}\raggedright
\texttt{data/derived/regression\_input.dta}\strut
\end{minipage} & \begin{minipage}[t]{0.19\columnwidth}\raggedright
All listed\strut
\end{minipage} & \begin{minipage}[t]{0.23\columnwidth}\raggedright
Combines multiple data sources, serves as input for Table 2, 3 and
Figure 5.\strut
\end{minipage} & \begin{minipage}[t]{0.21\columnwidth}\raggedright
Yes\strut
\end{minipage}\tabularnewline
\bottomrule
\end{longtable}

\hypertarget{computational-requirements}{%
\subsection{Computational
requirements}\label{computational-requirements}}

\begin{quote}
INSTRUCTIONS: In general, the specific computer code used to generate
the results in the article will be within the repository that also
contains this README. However, other computational requirements - shared
libraries or code packages, required software, specific computing
hardware - may be important, and is always useful, for the goal of
replication. Some example text follows.
\end{quote}

\begin{quote}
INSTRUCTIONS: We strongly suggest providing setup scripts that
install/set up the environment. Sample scripts for
\href{https://github.com/gslab-econ/template/blob/master/config/config_stata.do}{Stata},
\href{https://github.com/labordynamicsinstitute/paper-template/blob/master/programs/global-libraries.R}{R},
\href{https://github.com/labordynamicsinstitute/paper-template/blob/master/programs/packages.jl}{Julia}
are easy to set up and implement. Specific software may have more
sophisticated tools:
\href{https://pip.pypa.io/en/stable/user_guide/\#ensuring-repeatability}{Python},
\href{https://julia.quantecon.org/more_julia/tools_editors.html\#Package-Environments}{Julia}.
\end{quote}

\hypertarget{software-requirements}{%
\subsubsection{Software Requirements}\label{software-requirements}}

\begin{quote}
INSTRUCTIONS: List all of the software requirements, up to and including
any operating system requirements, for the entire set of code. It is
suggested to distribute most dependencies together with the replication
package if allowed, in particular if sourced from unversioned code
repositories, Github repos, and personal webpages. In all cases, list
the version \emph{you} used.
\end{quote}

\begin{itemize}
\tightlist
\item
  Stata (code was last run with version 15)

  \begin{itemize}
  \tightlist
  \item
    \texttt{estout} (as of 2018-05-12)
  \item
    \texttt{rdrobust} (as of 2019-01-05)
  \item
    the program ``\texttt{0\_setup.do}'' will install all dependencies
    locally, and should be run once.
  \end{itemize}
\item
  Python 3.6.4

  \begin{itemize}
  \tightlist
  \item
    \texttt{pandas} 0.24.2
  \item
    \texttt{numpy} 1.16.4
  \item
    the file ``\texttt{requirements.txt}'' lists these dependencies,
    please run ``\texttt{pip\ install\ -r\ requirements.txt}'' as the
    first step. See
    \url{https://pip.pypa.io/en/stable/user_guide/\#ensuring-repeatability}
    for further instructions on creating and using the
    ``\texttt{requirements.txt}'' file.
  \end{itemize}
\item
  Intel Fortran Compiler version 20200104
\item
  Matlab (code was run with Matlab Release 2018a)
\item
  R 3.4.3

  \begin{itemize}
  \tightlist
  \item
    \texttt{tidyr} (0.8.3)
  \item
    \texttt{rdrobust} (0.99.4)
  \item
    the file ``\texttt{0\_setup.R}'' will install all dependencies
    (latest version), and should be run once prior to running other
    programs.
  \end{itemize}
\end{itemize}

Portions of the code use bash scripting, which may require Linux.

Portions of the code use Powershell scripting, which may require Windows
10 or higher.

\hypertarget{controlled-randomness}{%
\subsubsection{Controlled Randomness}\label{controlled-randomness}}

\begin{quote}
INSTRUCTIONS: Some estimation code uses random numbers, almost always
provided by pseudorandom number generators (PRNGs). For reproducibility
purposes, these should be provided with a deterministic seed, so that
the sequence of numbers provided is the same for the original author and
any replicators. While this is not always possible, it is a requirement
by many journals' policies. The seed should be set once, and not use a
time-stamp. If using parallel processing, special care needs to be
taken. If using multiple programs in sequence, care must be taken on how
to call these programs, ideally from a main program, so that the
sequence is not altered.
\end{quote}

\begin{itemize}
\tightlist
\item[$\square$]
  Random seed is set at line \_\_\_\_\_ of program \_\_\_\_\_\_
\end{itemize}

\hypertarget{memory-and-runtime-requirements}{%
\subsubsection{Memory and Runtime
Requirements}\label{memory-and-runtime-requirements}}

\begin{quote}
INSTRUCTIONS: Memory and compute-time requirements may also be relevant
or even critical. Some example text follows. It may be useful to break
this out by Table/Figure/section of processing. For instance, some
estimation routines might run for weeks, but data prep and creating
figures might only take a few minutes.
\end{quote}

\hypertarget{summary}{%
\paragraph{Summary}\label{summary}}

Approximate time needed to reproduce the analyses on a standard (CURRENT
YEAR) desktop machine:

\begin{itemize}
\tightlist
\item[$\square$]
  \textless10 minutes
\item[$\square$]
  10-60 minutes
\item[$\square$]
  1-2 hours
\item[$\square$]
  2-8 hours
\item[$\square$]
  8-24 hours
\item[$\square$]
  1-3 days
\item[$\square$]
  3-14 days
\item[$\square$]
  \textgreater{} 14 days
\item[$\square$]
  Not feasible to run on a desktop machine, as described below.
\end{itemize}

\hypertarget{details}{%
\paragraph{Details}\label{details}}

The code was last run on a \textbf{4-core Intel-based laptop with MacOS
version 10.14.4}.

Portions of the code were last run on a \textbf{32-core Intel server
with 1024 GB of RAM, 12 TB of fast local storage}. Computation took 734
hours.

Portions of the code were last run on a \textbf{12-node AWS R3 cluster,
consuming 20,000 core-hours}.

\begin{quote}
INSTRUCTIONS: Identifiying hardware and OS can be obtained through a
variety of ways: Some of these details can be found as follows:

\begin{itemize}
\tightlist
\item
  (Windows) by right-clicking on ``This PC'' in File Explorer and
  choosing ``Properties''
\item
  (Mac) Apple-menu \textgreater{} ``About this Mac''
\item
  (Linux) see code in
  \href{https://github.com/AEADataEditor/replication-template/blob/master/tools/linux-system-info.sh}{tools/linux-system-info.sh}`
\end{itemize}
\end{quote}

\hypertarget{description-of-programscode}{%
\subsection{Description of
programs/code}\label{description-of-programscode}}

\begin{quote}
INSTRUCTIONS: Give a high-level overview of the program files and their
purpose. Remove redundant/ obsolete files from the Replication archive.
\end{quote}

\begin{itemize}
\tightlist
\item
  Programs in \texttt{programs/01\_dataprep} will extract and reformat
  all datasets referenced above. The file
  \texttt{programs/01\_dataprep/main.do} will run them all.
\item
  Programs in \texttt{programs/02\_analysis} generate all tables and
  figures in the main body of the article. The program
  \texttt{programs/02\_analysis/main.do} will run them all. Each program
  called from \texttt{main.do} identifies the table or figure it creates
  (e.g., \texttt{05\_table5.do}). Output files are called appropriate
  names (\texttt{table5.tex}, \texttt{figure12.png}) and should be easy
  to correlate with the manuscript.
\item
  Programs in \texttt{programs/03\_appendix} will generate all tables
  and figures in the online appendix. The program
  \texttt{programs/03\_appendix/main-appendix.do} will run them all.
\item
  Ado files have been stored in \texttt{programs/ado} and the
  \texttt{main.do} files set the ADO directories appropriately.
\item
  The program \texttt{programs/00\_setup.do} will populate the
  \texttt{programs/ado} directory with updated ado packages, but for
  purposes of exact reproduction, this is not needed. The file
  \texttt{programs/00\_setup.log} identifies the versions as they were
  last updated.
\item
  The program \texttt{programs/config.do} contains parameters used by
  all programs, including a random seed. Note that the random seed is
  set once for each of the two sequences (in \texttt{02\_analysis} and
  \texttt{03\_appendix}). If running in any order other than the one
  outlined below, your results may differ.
\end{itemize}

\hypertarget{optional-but-recommended-license-for-code}{%
\subsubsection{(Optional, but recommended) License for
Code}\label{optional-but-recommended-license-for-code}}

\begin{quote}
INSTRUCTIONS: Most journal repositories provide for a default license,
but do not impose a specific license. Authors should actively select a
license. This should be provided in a LICENSE.txt file, separately from
the README, possibly combined with the license for any data provided.
Some code may be subject to inherited license requirements, i.e., the
original code author may allow for redistribution only if the code is
licensed under specific rules - authors should check with their sources.
For instance, some code authors require that their article describing
the econometrics of the package be cited. Licensing can be complex. Some
non-legal guidance may be found
\href{https://social-science-data-editors.github.io/guidance/Licensing_guidance.html}{here}.
\end{quote}

The code is licensed under a MIT/BSD/GPL {[}choose one!{]} license. See
\href{https://social-science-data-editors.github.io/template_README/LICENSE.txt}{LICENSE.txt}
for details.

\hypertarget{instructions-to-replicators}{%
\subsection{Instructions to
Replicators}\label{instructions-to-replicators}}

\begin{quote}
INSTRUCTIONS: The first two sections ensure that the data and software
necessary to conduct the replication have been collected. This section
then describes a human-readable instruction to conduct the replication.
This may be simple, or may involve many complicated steps. It should be
a simple list, no excess prose. Strict linear sequence. If more than 4-5
manual steps, please wrap a main program/Makefile around them, in
logical sequences. Examples follow.
\end{quote}

\begin{itemize}
\tightlist
\item
  Edit \texttt{programs/config.do} to adjust the default path
\item
  Run \texttt{programs/00\_setup.do} once on a new system to set up the
  working environment.
\item
  Download the data files referenced above. Each should be stored in the
  prepared subdirectories of \texttt{data/}, in the format that you
  download them in. Do not unzip. Scripts are provided in each directory
  to download the public-use files. Confidential data files requested as
  part of your FSRDC project will appear in the \texttt{/data} folder.
  No further action is needed on the replicator's part.
\item
  Run \texttt{programs/01\_main.do} to run all steps in sequence.
\end{itemize}

\hypertarget{details-1}{%
\subsubsection{Details}\label{details-1}}

\begin{itemize}
\tightlist
\item
  \texttt{programs/00\_setup.do}: will create all output directories,
  install needed ado packages.

  \begin{itemize}
  \tightlist
  \item
    If wishing to update the ado packages used by this archive, change
    the parameter \texttt{update\_ado} to \texttt{yes}. However, this is
    not needed to successfully reproduce the manuscript tables.
  \end{itemize}
\item
  \texttt{programs/01\_dataprep}:

  \begin{itemize}
  \tightlist
  \item
    These programs were last run at various times in 2018.
  \item
    Order does not matter, all programs can be run in parallel, if
    needed.
  \item
    A \texttt{programs/01\_dataprep/main.do} will run them all in
    sequence, which should take about 2 hours.
  \end{itemize}
\item
  \texttt{programs/02\_analysis/main.do}.

  \begin{itemize}
  \tightlist
  \item
    If running programs individually, note that ORDER IS IMPORTANT.
  \item
    The programs were last run top to bottom on July 4, 2019.
  \end{itemize}
\item
  \texttt{programs/03\_appendix/main-appendix.do}. The programs were
  last run top to bottom on July 4, 2019.
\item
  Figure 1: The figure can be reproduced using the data provided in the
  folder ``2\_data/data\_map'', and ArcGIS Desktop (Version 10.7.1) by
  following these (manual) instructions:

  \begin{itemize}
  \tightlist
  \item
    Create a new map document in ArcGIS ArcMap, browse to the folder
    ``2\_data/data\_map'' in the ``Catalog'', with files
    ``provinceborders.shp'', ``lakes.shp'', and ``cities.shp''.
  \item
    Drop the files listed above onto the new map, creating three
    separate layers. Order them with ``lakes'' in the top layer and
    ``cities'' in the bottom layer.
  \item
    Right-click on the cities file, in properties choose the variable
    ``health''\ldots{} (more details)
  \end{itemize}
\end{itemize}

\hypertarget{list-of-tables-and-programs}{%
\subsection{List of tables and
programs}\label{list-of-tables-and-programs}}

\begin{quote}
INSTRUCTIONS: Your programs should clearly identify the tables and
figures as they appear in the manuscript, by number. Sometimes, this may
be obvious, e.g.~a program called ``\texttt{table1.do}'' generates a
file called \texttt{table1.png}. Sometimes, mnemonics are used, and a
mapping is necessary. In all circumstances, provide a list of tables and
figures, identifying the program (and possibly the line number) where a
figure is created.

NOTE: If the public repository is incomplete, because not all data can
be provided, as described in the data section, then the list of tables
should clearly indicate which tables, figures, and in-text numbers can
be reproduced with the public material provided.
\end{quote}

The provided code reproduces:

\begin{itemize}
\tightlist
\item[$\square$]
  All numbers provided in text in the paper
\item[$\square$]
  All tables and figures in the paper
\item[$\square$]
  Selected tables and figures in the paper, as explained and justified
  below.
\end{itemize}

\begin{longtable}[]{@{}lllll@{}}
\toprule
\begin{minipage}[b]{0.13\columnwidth}\raggedright
Figure/Table \#\strut
\end{minipage} & \begin{minipage}[b]{0.18\columnwidth}\raggedright
Program\strut
\end{minipage} & \begin{minipage}[b]{0.09\columnwidth}\raggedright
Line Number\strut
\end{minipage} & \begin{minipage}[b]{0.23\columnwidth}\raggedright
Output file\strut
\end{minipage} & \begin{minipage}[b]{0.23\columnwidth}\raggedright
Note\strut
\end{minipage}\tabularnewline
\midrule
\endhead
\begin{minipage}[t]{0.13\columnwidth}\raggedright
Table 1\strut
\end{minipage} & \begin{minipage}[t]{0.18\columnwidth}\raggedright
02\_analysis/table1.do\strut
\end{minipage} & \begin{minipage}[t]{0.09\columnwidth}\raggedright
\strut
\end{minipage} & \begin{minipage}[t]{0.23\columnwidth}\raggedright
summarystats.csv\strut
\end{minipage} & \begin{minipage}[t]{0.23\columnwidth}\raggedright
\strut
\end{minipage}\tabularnewline
\begin{minipage}[t]{0.13\columnwidth}\raggedright
Table 2\strut
\end{minipage} & \begin{minipage}[t]{0.18\columnwidth}\raggedright
02\_analysis/table2and3.do\strut
\end{minipage} & \begin{minipage}[t]{0.09\columnwidth}\raggedright
15\strut
\end{minipage} & \begin{minipage}[t]{0.23\columnwidth}\raggedright
table2.csv\strut
\end{minipage} & \begin{minipage}[t]{0.23\columnwidth}\raggedright
\strut
\end{minipage}\tabularnewline
\begin{minipage}[t]{0.13\columnwidth}\raggedright
Table 3\strut
\end{minipage} & \begin{minipage}[t]{0.18\columnwidth}\raggedright
02\_analysis/table2and3.do\strut
\end{minipage} & \begin{minipage}[t]{0.09\columnwidth}\raggedright
145\strut
\end{minipage} & \begin{minipage}[t]{0.23\columnwidth}\raggedright
table3.csv\strut
\end{minipage} & \begin{minipage}[t]{0.23\columnwidth}\raggedright
\strut
\end{minipage}\tabularnewline
\begin{minipage}[t]{0.13\columnwidth}\raggedright
Figure 1\strut
\end{minipage} & \begin{minipage}[t]{0.18\columnwidth}\raggedright
n.a. (no data)\strut
\end{minipage} & \begin{minipage}[t]{0.09\columnwidth}\raggedright
\strut
\end{minipage} & \begin{minipage}[t]{0.23\columnwidth}\raggedright
\strut
\end{minipage} & \begin{minipage}[t]{0.23\columnwidth}\raggedright
Source: Herodus (2011)\strut
\end{minipage}\tabularnewline
\begin{minipage}[t]{0.13\columnwidth}\raggedright
Figure 2\strut
\end{minipage} & \begin{minipage}[t]{0.18\columnwidth}\raggedright
02\_analysis/fig2.do\strut
\end{minipage} & \begin{minipage}[t]{0.09\columnwidth}\raggedright
\strut
\end{minipage} & \begin{minipage}[t]{0.23\columnwidth}\raggedright
figure2.png\strut
\end{minipage} & \begin{minipage}[t]{0.23\columnwidth}\raggedright
\strut
\end{minipage}\tabularnewline
\begin{minipage}[t]{0.13\columnwidth}\raggedright
Figure 3\strut
\end{minipage} & \begin{minipage}[t]{0.18\columnwidth}\raggedright
02\_analysis/fig3.do\strut
\end{minipage} & \begin{minipage}[t]{0.09\columnwidth}\raggedright
\strut
\end{minipage} & \begin{minipage}[t]{0.23\columnwidth}\raggedright
figure-robustness.png\strut
\end{minipage} & \begin{minipage}[t]{0.23\columnwidth}\raggedright
Requires confidential data\strut
\end{minipage}\tabularnewline
\bottomrule
\end{longtable}

\hypertarget{references}{%
\subsection{References}\label{references}}

\begin{quote}
INSTRUCTIONS: As in any scientific manuscript, you should have proper
references. For instance, in this sample README, we cited ``Ruggles et
al, 2019'' and ``DESE, 2019'' in a Data Availability Statement. The
reference should thus be listed here, in the style of your journal:
\end{quote}

Steven Ruggles, Steven M. Manson, Tracy A. Kugler, David A. Haynes II,
David C. Van Riper, and Maryia Bakhtsiyarava. 2018. ``IPUMS Terra:
Integrated Data on Population and Environment: Version 2
{[}dataset{]}.'' Minneapolis, MN: \emph{Minnesota Population Center,
IPUMS}. https://doi.org/10.18128/D090.V2

Department of Elementary and Secondary Education (DESE), 2019. ``Student
outcomes database {[}dataset{]}'' \emph{Massachusetts Department of
Elementary and Secondary Education (DESE)}. Accessed January 15, 2019.

U.S. Bureau of Economic Analysis (BEA). 2016. ``Table 30:''Economic
Profile by County, 1969-2016.'' (accessed Sept 1, 2017).

Inglehart, R., C. Haerpfer, A. Moreno, C. Welzel, K. Kizilova, J.
Diez-Medrano, M. Lagos, P. Norris, E. Ponarin \& B. Puranen et
al.~(eds.). 2014. World Values Survey: Round Six - Country-Pooled
Datafile Version:
http://www.worldvaluessurvey.org/WVSDocumentationWV6.jsp. Madrid: JD
Systems Institute.

\begin{center}\rule{0.5\linewidth}{0.5pt}\end{center}

\hypertarget{acknowledgements}{%
\subsection{Acknowledgements}\label{acknowledgements}}

Some content on this page was copied from
\href{https://www.hindawi.com/research.data/\#statement.templates}{Hindawi}.
Other content was adapted from
\href{https://doi.org/10.1093/restud/rdw057}{Fort (2016)}, Supplementary
data, with the author's permission.

\end{document}
